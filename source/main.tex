% !TeX document-id = {2dbf9b71-bbba-4dd9-bbd6-409e1a74ac36}
% !TeX TXS-program:compile = txs:///pdflatex/[--shell-escape] | txs:///biber | txs:///pdflatex/[--shell-escape]
\PassOptionsToPackage{svgnames}{xcolor}
\documentclass[spanish, utf8,handout]{beamer} % 
\usepackage[T1]{fontenc}
\usepackage{mathpazo}
\usepackage[spanish]{babel}
\usepackage{amsmath,mathrsfs,amsfonts,amsthm}
\usepackage{tikz}
\usepackage{tkz-graph}
\usepackage{adjustbox}
\usepackage{caption}

%\usepackage{sagetex}
%\usepackage{pythontex}
%\usepackage[figurename=Figura]{caption}
\usepackage[backend=biber,style=numeric, defernumbers=true, sorting=ynt,maxbibnames=4,maxcitenames=4]{biblatex}
\addbibresource{bibliography/reference.bib}
\renewcommand{\figurename}{Figura}
\renewcommand{\contentsname}{Índice analítico}
\theoremstyle{definition}
\newtheorem{remark}{Observación}

\usetheme{CambridgeUS}
\usecolortheme{dolphin}
\useinnertheme{rectangles}
%\useoutertheme[hooks]{tree}

\usefonttheme[onlymath]{serif}

%\setbeamercovered{transparent}
\setbeamercovered{dynamic}

\setbeamertemplate{background canvas}{\includegraphics[height=\paperheight]{background}}
\setbeamertemplate{footline}[frame number]{}
\setbeamertemplate{navigation symbols}{}
\setbeamertemplate{footline}{}
\setbeamertemplate{headline}{}
\setbeamertemplate{blocks}[rounded][shadow=false] 

\title[Teorema de los cuatro colores]{\Huge\sffamily El teorema de los cuatro colores}
\subtitle{Introducción a la matemática discreta CM -- 254}

\author[Grupo N$^\circ6$]{%
	\texorpdfstring{%
		\begin{columns}
			\column{.3\linewidth}
			\centering
			C. Aznarán Laos %\inst{1,2}
			\column{.3\linewidth}
			\centering
			F. Cruz Ordoñez %\inst{1,2}
		\end{columns}
		\vspace{12pt}
		\begin{columns}
			\column{.3\linewidth}
			\centering
			G. Quiroz Gómez %\inst{1,2}
			\column{.3\linewidth}
			\centering
			J. Navío Torres %\inst{1,2}
		\end{columns}
	}
	{Author 1, Author 2, Author 3}
}

\institute[FC -- UNI]{\large%\inst{1}
	Facultad de Ciencias \and%\inst{2}
	Universidad Nacional de Ingeniería
}
\date{18 de junio del 2018}

\graphicspath{{images/}}

\AtBeginSubsection[]
{
	\begin{frame}<beamer>
	\frametitle{\contentsname}
	\tableofcontents[
	currentsection,
	sectionstyle=show/show,
	subsectionstyle=show/shaded/hide%-show/shaded/hide
	]
\end{frame}
}

\begin{document}

\begin{frame}[plain]
\maketitle
\end{frame}

\begin{frame}{\contentsname}\transblindsvertical
\tableofcontents
\end{frame}

\section{Introducción}

\subsection{El problema de los cuatro colores}

\begin{frame}\transblindsvertical
\frametitle{\insertsubsection}

\begin{alertblock}{Pregunta} % 1852 Francis Guthrie
¿Es posible colorear cualquier mapa geográfico plano usando solamente \emph{\color{DarkBlue}cuatro colores}, de modo que dos países con frontera común tengan colores distintos? 
\end{alertblock}

\

\visible<2>{
\begin{minipage}[c]{6cm}
	\begin{figure}[H]
		\includegraphics[width=4.2cm]{mapa}
		\caption{Mapa político coloreado}
	\end{figure}
\end{minipage}%\vspace{-1cm}
\begin{minipage}[c]{6cm}
	\begin{definition}[Mapa conexo]
	Un mapa es conexo\footnote{De una sola pieza.} y cada una \linebreak de sus regiones también es conexa.
	\end{definition}
\end{minipage}
}
\end{frame}

\begin{frame}{\insertsubsection}
\begin{remark}{}
Dos regiones {\color{red}no pueden tocarse solo en un punto}, y así, se pueden ignorar regiones con una única línea frontera.
\end{remark}

\visible<2>{
\begin{figure}[H]
	\centering
	\includegraphics[height=3.3cm]{frontera}
	\caption{Distinciones de frontera de un mapa.}
\end{figure}

Es un problema topológico: no importa la forma de las regiones, sino como están colocadas unas respecto a otras.
}
\end{frame}

{
	\usebackgroundtemplate{\centering\includegraphics[width=\paperwidth]{timeline}}
	\begin{frame}[plain]
\end{frame}
}

\subsection{Algunas fechas importantes}

\begin{frame}\transblindsvertical
\frametitle{\insertsubsection}

\begin{itemize}
	\item {\color{DarkBlue}1852}: Francis Guthrie plantea el problema a su hermano Frederick y éste a Augustus de Morgan.
	
	\item  {\color{DarkBlue}1878}: Arthur Cayley publica el enunciado de la conjetura.
	
	\item  {\color{DarkBlue}1879}: Sir Alfred Bray Kempe publica su ``demostración''.
	
	\item  {\color{DarkBlue}1913}: George Birkhoff introduce la noción de configuración reducible.
	
	\item  {\color{DarkBlue}1960}: Se introduce el llamado método de descarga.
	
	\item  {\color{DarkBlue}1969}: Avances de Heinrich Heesch en reducibilidad y obtención de conjuntos inevitables de configuraciones.
	
	\item {\color{DarkBlue}1976}: Ken Appel y Wolfgang Haken prueban con ayuda de un ordenador que sus 1.482 configuraciones son reducibles (50 días de cálculo).
	
	\item  {\color{DarkBlue}1996}: N. Robertson, D.P. Sanders, P. Seymour y R. Thomas mejoran la demostración con ayuda de ordenador (solo 633 configuraciones) y automatizan la prueba de la inevitabilidad.
\end{itemize}   
\end{frame}

\section{El ``camino'' hacia la demostración}

\subsection{La formulación de la conjetura}

\begin{frame}\transblindsvertical
\frametitle{\insertsubsection}

\begin{exampleblock}{Francis Guthrie (1839-1899)}
Abogado y botánico, observa que puede colorear un mapa complejo de los cantones de Inglaterra con 4 colores. En 1852, enuncia el problema a su hermano Frederick (University College London) y a éste a Augustus de Morgan. Francis Guthrie observa que 3 colores no son suficientes, con el diagrama crítico:
\end{exampleblock}
\visible<2>{
\begin{figure}[H]
\centering
\includegraphics[scale=0.3]{critico}
\caption{Diagrama Crítico.}
\end{figure}
}
\end{frame}

\begin{frame}\transblindsvertical
\frametitle{\insertsection}

\begin{alertblock}{Difusión del teorema}
Augustus de Morgan (1806-1871) estaba muy interesado en la conjetura de los cuatro colores y lo difundió entre sus colegas. Una de las primeras personas con las que ``habló'' fue con el matemático y físico irlandés Sir William Rowan Hamilton (1805-1865), que no compartía el interés de De Morgan por el problema. Le escribe una carta el 23 de octubre de 1852.
\end{alertblock}

\visible<2>{
\begin{alertblock}{Respuesta de Hamilton}
Cuatro días después, Hamilton le contesta: ``I am not likely to attempt your ``quaternion'' of colours very soon''.
\end{alertblock}
} 
\end{frame}

\begin{frame}[allowframebreaks]
\frametitle{Definiciones previas}

\begin{definition}[Número cromático] 
Sea $G=(V,E)$ un grafo y sea $k$ un número natural. Una aplicación $c\colon V\to \{1,2,\ldots k\}$ se llama \emph{\color{DarkBlue}coloración del grafo} $G$ si $c(x)\neq c(y)$ se cumple para cada rama $\{x,y\}\in E$. \linebreak El \emph{\color{DarkBlue}número cromático} de $G$, denotado por $\chi(G)$, es el \emph{\color{red}mínimo valor} de $k$ para el cual existe una coloración $c\colon V(G)\to\{1,2\ldots,k\}$.
\end{definition}

\begin{definition}[Grafo Dual]
Sea $G=(V,E)$ un grafo planar con un dibujo planar fijo. Denotamos por $\mathcal{F}$ el conjunto de caras de $G$. Definimos un grafo, con posibles lazos y ramas múltiples, como $(\mathcal{F},E,\varepsilon)$, donde $\varepsilon$ se define como $\varepsilon(e)=\{F_i,F_j\}$ siempre que la rama $e$ sea una frontera común de las caras $F_i$ y $F_j$.

Este grafo $\left(\mathcal{F},E,\varepsilon\right)$ se le llama el dual de $G$ y se denota por $G^{\ast}$.	
\end{definition}


\begin{example}[Grafos Duales]
Para construir una gráfica dual de un grafo plano $G$ se debe colocar un vértice dentro de cada región de $G$ e incluir la región infinita de $G$. Para cada arista compartida por las $2$ regiones, se debe dibujar una arista que conecte a los vértices dentro de estas regiones y para cada arista que se recorre $2$ veces en el camino cerrado alrededor de las aristas de una región se dibuja un lazo en el vértice de la región. 
\end{example}

\begin{figure}[H]
	\captionsetup{justification=centering,margin=0.5cm}
	\centering
	\begin{minipage}{.5\textwidth}
		\centering
		\includegraphics[width=4cm]{example1}
		\caption{Grafo $G$.}
	\end{minipage}%
	\begin{minipage}{0.5\textwidth}
		\centering
		\includegraphics[width=4cm]{example2}
		\caption{Grafo $G$ y su dual $G^{\ast}$.}
	\end{minipage}
\end{figure}

\begin{example}[Grafos duales]
	Sea $G=(V,E)$ un grafo plano, se llama grafo dual de $G$ y se denota por $G^{\ast}$, aquel construido de la siguiente manera:
	
	\begin{enumerate}
		\item Se elige un punto $v_i$ en cada cara $F_i$ de $G$. Estos puntos son los vértices de $G^{\ast}$.
		
		\item Por cada arista $e\in E$ se traza una línea $e^{\ast}$ que atraviesa únicamente la arista $e$, y se unen los vértices $v_i$ pertenecientes a las caras adjuntas a $e$. Estas líneas son las aristas de $G^{\ast}$. A continuación se ilustra este procedimiento de construcción con un ejemplo:
	\end{enumerate}
\end{example}

\begin{figure}[H]
	\captionsetup{justification=centering,margin=0.5cm}
	\centering
	\includegraphics{example3}
	\caption{Grafo planar $G$ y	su grafo dual $G^{\prime}=G^{\ast}$.}
\end{figure}

\begin{definition}[Polinomio cromático]
Sea $G=(V,E)$ un grafo planar y $P(G,k)$ el número de vértices coloreados. El polinomio cromático cuenta el número de maneras que puede ser coloreado un grafo usando no más de número dado de colores.
\end{definition}

\begin{remark}
El polinomio cromático incluye al menos tanta información sobre la colorabilidad de $G$ como el número cromático. De hecho, $\chi$ es el entero positivo más pequeño que no es una raíz del polinomio cromático

\begin{equation*}
\chi(G)=\min\{k\colon P(G,k)>0\}.
\end{equation*}
\end{remark}

\end{frame}

\subsection{La primera ``demostración'': las cadenas de \citeauthor{kempe}}

\begin{frame}\transblindsvertical
\frametitle{\insertsubsection}

Kempe se interesa por el problema de los cuatro colores tras la pregunta de Cayley en la London Mathematical Society.

\

\visible<2>{
En junio de 1879 obtiene su solución del teorema de los cuatro colores y lo publica en el Amer. Journal of Maths. En 1880, publica unas versiones. simplificadas de su prueba, donde corrige algunas erratas de su prueba original, pero deja intacto el error fatal.
}

\end{frame}

\begin{frame}
\frametitle{El algoritmo de las cadenas de Kempe}

\begin{definition}[Cadena de Kempe]
Sea $G$ un grafo planar cuyos vértices han sido coloreados apropiadamente y suponga $v\in V(G)$ es coloreado $C_1$. Definimos la \emph{cadena de Kempe} $C_1C_2$ que contiene a $v$ para ser el componente conexa maximal de $G$ que
\begin{enumerate}%[label=\arabic*]
	\item Contiene a $v$, y
	\item Contenga solo vértices que son coloreados con elementos desde $(C_1C_2)$.
\end{enumerate}
\end{definition}
\end{frame}

\subsection{Heawood y el error fatal de \citeauthor{kempe}}

\begin{frame}[allowframebreaks]
\frametitle{\insertsubsection}

\begin{example}[Grafo de Errera -- Contrajemplo]
Es un grafo planar de $17$ vértices y $45$ aristas que enreda las cadenas de Kempe en el algoritmo de Kempe y proporciona así un ejemplo de cómo falla la supuesta demostración de Kempe del teorema de cuatro colores.
\end{example}

\begin{figure}[H]
	\centering
	\scalebox{.3}{\begin{tikzpicture}
	\SetGraphUnit{2}
	\GraphInit[vstyle=Classic]
	\SetVertexNoLabel
	\foreach \X/\Y [count=\Z] in {4/ 13, 4/ 6, -0.2/ 5, 2/ 2.2, -6/ -1, 6/ 2.2, 
	9.5/ 1.6, 2.7/ 5, 4.8/ 7.8, 4/ 10.4, -1.5/ 1.6, 14/ -1, 4/ 3.6, 
	5.3/ 5, 3.2/ 7.8, 4/ 0.3, 8.1/ 5}
	{\Vertex[x=\X,y=\Y]{v\Z}}
	\Edges(v2,v8,v13,v14,v2)
	\Edges(v5,v12,v16,v5)
    \Edges(v5,v11,v16,v7,v12)
	\Edges(v11,v4,v16,v6,v7)
	\Edges(v4,v8,v15,v10,v9,v14,v6)
	\Edges(v4,v13,v6,v4)
	\Edges(v13,v2,v15,v9,v2)
	\Edges(v3,v10,v17)
	\Edges(v11,v15,v3)
	\Edges(v7,v9,v17)
	\Edges(v5,v3,v11)
	\Edges(v12,v17,v7)
	\Edges(v5,v1,v12)
	\Edges(v3,v1,v17)
	\Edges(v1,v10)
	\Edges(v11,v8)
	\Edges(v7,v14)
\end{tikzpicture}}
	\caption{Grafo de Errera.}
\end{figure}

\begin{example}[Grafo de Kittel -- Contrajemplo]
Es un grafo planar de $23$ vértices y $63$ aristas que enreda las cadenas de Kempe en el algoritmo de Kempe y proporciona así un ejemplo de cómo falla la supuesta demostración de Kempe del teorema de cuatro colores.
\end{example}

\begin{figure}[H]
	\centering
	\scalebox{.25}{\input{./images/kittel.tex}}
	\caption{Grafo de Kittel.}
\end{figure}

\begin{example}[Grafo de Soifer -- Contrajemplo]
Es un grafo planar de $9$ vértices y $20$ aristas que enreda las cadenas de Kempe en el algoritmo de Kempe y proporciona así un ejemplo de cómo falla la supuesta demostración de Kempe del teorema de cuatro colores.
\end{example}

\begin{figure}[H]
	\centering
	\scalebox{0.6}{\begin{tikzpicture}[scale=2]
	\SetGraphUnit{2}
	\GraphInit[vstyle=Classic]
	\SetVertexNoLabel
	\foreach \X/\Y [count=\Z] in {5 /0, 2.5/ 2.35, 0/ 0, 3.2/ 3, 1.8/ 1.7, 3.2/ 1.7,
		1.4/ 0.6, 1.8/ 3, 2.5/ 1.05}
	{\Vertex[x=\X,y=\Y]{v\Z}}
	\Edges(v3,v8,v4,v2,v8,v5,v3,v7,v5,v9,v2,v6,v9,v7,v1,v9)
	\Edges(v1,v6,v4,v1)
	\Edges(v5,v2)
	\Edges(v3,v1)
\end{tikzpicture}}
	\caption{Grafo de Soifer.}
\end{figure}

\begin{example}[Grafo de Fritsch -- Contrajemplo]
Es un grafo planar de $9$ vértices y $21$ aristas que enreda las cadenas de Kempe en el algoritmo de Kempe y proporciona así un ejemplo de cómo falla la supuesta demostración de Kempe del teorema de cuatro colores.
\end{example}

\begin{figure}[H]
	\centering
	\scalebox{0.45}{\begin{tikzpicture}[scale=6]
	\SetGraphUnit{2}
	\GraphInit[vstyle=Classic]
	\SetVertexNoLabel
	\foreach \X/\Y [count=\Z] in {0.866/ -0.5, 0/ 1, 0.263/ -0.235, 0.188/ 0.0294, -0.866/ -0.5, 0/ -0.323,  -0.263/ -0.235, -0.188/ 0.029, 0/-0.147
	}
	{\Vertex[x=\X,y=\Y]{v\Z}}
	\Edges(v5,v2,v1,v5,v6,v1)
	\Edges(v5,v7,v9,v4,v3,v6,v7,v3,v9,v8,v7)	
	\Edges(v2,v9)
	\Edges(v5,v8,v2,v4,v1,v3)
\end{tikzpicture}}
	\caption{Grafo de Fritsch.}
\end{figure}


\begin{example}[Grafo de Poussin -- Contrajemplo]
Es un grafo planar de $15$ vértices y $39$ aristas que enreda las cadenas de Kempe en el algoritmo de Kempe y proporciona así un ejemplo de cómo falla la supuesta demostración de Kempe del teorema de cuatro colores.
\end{example}

\begin{figure}[H]
	\centering
	\scalebox{0.6}{\begin{tikzpicture}[scale=6.5]
	\SetGraphUnit{2}
	\GraphInit[vstyle=Classic]
	\SetVertexNoLabel
	\foreach \X/\Y [count=\Z] in {2.21/ 5.19, 2.19/ 5.4, 2.27/ 5.28, 1.98/ 5.57,  2.42/ 5.54, 2.11/ 5.28, 2.19/ 5.77, 1.61/ 5.07, 2.32/ 5.44, 2.07/ 5.44, 2.85/ 5.07, 2.61/ 5.19, 1.86/ 5.15, 2.19/ 5.53, 2.19/ 6.02}
	{\Vertex[x=\X,y=\Y]{v\Z}}
	\Edges(v8,v15,v11,v8,v13,v4,v7,v5,v12,v1,v13)
	\Edges(v11,v13)
	\Edges(v11,v1)
	\Edges(v11,v12)
	\Edges(v8,v4)
	\Edges(v4,v15)
	\Edges(v15,v7,v14,v4,v10,v13,v6,v1,v3)
	\Edges(v12,v3,v6,v10,v14,v9,v3,v2)
	\Edges(v2,v6)
	\Edges(v2,v10)
	\Edges(v2,v14)
	\Edges(v2,v9,v12)
	\Edges(v7,v9)
	\Edges(v7,v12)
	\Edges(v15,v5,v11)
\end{tikzpicture}}
	\caption{Grafo de Poussin.}
\end{figure}

\begin{remark}[Polinomio cromático del grafo de Errera]
A
\end{remark}
\end{frame}

\begin{frame}\transblindsvertical
\frametitle{\insertsubsection}

\begin{theorem}[Fórmula de Euler para mapas]
\begin{equation*}
\#\text{caras} - \#\text{aristas} + \#\text{vértices} = 2.	
\end{equation*}
\end{theorem}

\visible<2>{
\begin{figure}[H]
\centering
\includegraphics[height=5cm]{poliedro}
\caption{Grafos de cada uno de los cinco sólidos platónicos.}
\end{figure}
}

\end{frame}

\subsection{Idea clave: La reducibilidad de mapas de \citeauthor{birkhoff}}

\begin{frame}
\frametitle{\insertsubsection}

\begin{minipage}[c]{4.7cm}
\begin{theorem}[\citeauthor{birkhoff}]\textnormal
Solo una de las siguientes afirmaciones es verdadera:
\begin{itemize}%[wide=\parindent]%leftmargin=0.5em],
	\item \small{La conjetura de los cuatro colores puede ser falsa.}
	\item \small{Es posible hallar una colección finita de configuraciones reducibles \linebreak tal que cualquier mapa planar debe contener uno \linebreak de ellos.}
	\item \small{La conjetura de cuatro colores puede ser cierta, pero pueden \linebreak requerirse métodos más complicados para una prueba.}
\end{itemize}
\end{theorem}
\end{minipage}
\begin{minipage}[c]{6.8cm}
\begin{figure}[H]
\centering
\includegraphics[width=6.6cm]{birkhoff}
\caption{Reducibilidad de mapas.}
\end{figure}
\end{minipage}
\end{frame}

\subsection{El método de descarga de \citeauthor{appel}}

\subsection{Una nueva demostración de \citeauthor{robertson}}

\section{Aplicaciones}

\subsection{El juego Hex}

\begin{frame}\transblindsvertical
\frametitle{\insertsubsection\footnote{Redescubierto en Priceton por John Nash en 1948.}}

\begin{columns}[t]
	\begin{column}{.4\textwidth}
			\begin{figure}[ht]
				%\centering
				\scalebox{.32}{% https://tex.stackexchange.com/questions/141911/drawing-hex-boards
\begin{tikzpicture}[rotate=90]
	%\showmynodenames%%<-- must precede the creation of the board if you want to see node names
	\drawhexboard{10}
	\draw[my hex path] ($(H1;1.corner 1)-(0,1cm)$) -- (H1;1.corner 1);
	\drawhexpath(1;1){1,2,3,4}
	\drawhexpath(2;2){6,5}
	\drawhexpath(3;2){1,2,3,4,5,6}
	\drawhexpath(3;1){2,1,6,5}
	\drawhexpath(4;1){1,2,3}
	\drawhexpath(5;2){1,2,3,4,5}
	
	\draw[my hex path] ($(H10;10.corner 3)+(1cm,0)$) -- (H10;10.corner 3);
	\drawhexpath(10;10){3,2,1,6,5}
	\drawhexpath(10;9){3,4}
	\drawhexpath(11;8){2,3,4}
	\drawhexpath(12;8){6,5,4,3,2}
\end{tikzpicture}}
				\caption{Patrón hexagonal.}
			\end{figure}%\adjincludegraphics[width=.8\linewidth,valign=t]{diagrama.jpg}
	\end{column}
	\begin{column}{.38\textwidth}
		\begin{flushleft}
		``El juego se basa en la simple propiedad geométrica de una superficie plana que dos líneas dentro de un cuadrado conectan cada una un par de lados opuestos deben cruzarse''.
		\end{flushleft}
		\		
		-- Piet Hein
	\end{column}
\end{columns}

\vspace*{5pt}

\begin{itemize}
	\item Reglas: Los jugadores se turnan para ocupar una celda y la otra para formar una cadena que conecta sus dos lados opuestos, y por lo tanto impide que la otra conecte sus lados, gana.
\end{itemize}
\end{frame}

%\begin{frame}\transblindsvertical
%\frametitle{\insertsubsection}
%\begin{tabular}{p{.3\textwidth} p{.7\textwidth}}
%	\adjincludegraphics[width=.8\linewidth,valign=t]{diagrama.jpg}
%	&
%	\raggedright\arraybackslash\textbf{``The problem of distinguishing prime numbers from composites, and of resolving composite numbers into their prime factors, is one of the most important and useful in all of arithmetic."}
%	
%	\hfill-- Carl Friedrich Gauss
%\end{tabular}
%
%\vspace*{10pt}
%
%\begin{itemize}
%	\item Pollard's $p-1$ algorithm (1974)
%	\vspace*{10pt}
%	\item Dixon's Random Squares Algorithm (1981)
%	\vspace*{10pt}
%	\item Quadratic Sieve (QS): Pomerance (1981)
%	\vspace*{10pt}
%	\item Williams' $p+1$ method (1982)
%\end{itemize}
%\end{frame}

%\section{Conclusiones}
%
%\subsection{Importancia del teorema para los matemáticos}

\begin{frame}
\frametitle{Agradecimientos}

\begin{center}\Large
	¡Muchas gracias!
\end{center}

Colaboradores:

\begin{enumerate}
	\item Elaboración de la línea de tiempo: José Navío.
	\item Tipografía en \LaTeX{}: Franss Cruz y Oromion.
	\item Explicación del contenido matemático: Gabriel Quiroz.
	\item Esquema de la exposición: MSc. Fidel Jara Huanca.
\end{enumerate}

\

{
\color{DarkBlue}
Presentación disponible en:
}
\begin{center}
\href{https://github.com/carlosal1015/4colores}{\includegraphics[width=2.5cm]{Octocat.png}}
\end{center}
\hfill
\begin{flushright}
Dudas, sugerencias o preguntas a

\href{mailto:caznaranl@uni.pe}{caznaranl\MVAt uni.pe}
\end{flushright}
\end{frame}

\begin{frame}[allowframebreaks]\transblindsvertical
\frametitle{Referencias}

\begin{itemize}
	\item Libros
	\nocite{*}
	\printbibliography[heading=none,keyword=book]
	
	\item Artículos matemáticos
	\printbibliography[heading=none,keyword=paper]
	
	\item Sitios web
	\printbibliography[heading=none,keyword=online]
\end{itemize}
\end{frame}

\end{document}

GraphData["ErreraGraph"] // InputForm

Graph[{1,2,3,4,5,6,7,8,9,10,11,12,13,14,15,16,17},
{Null,SparseArray[Automatic,{17,17},0,
	{1,{{0,5,10,15,20,25,30,36,41,47,52,58,63,68,73,79,85,90},
			{{3},{5},{10},{12},{17},{8},{9},{13},{14},{15},{1},{5},{10},{11},
				{15},{6},{8},{11},{13},{16},{1},{3},{11},{12},{16},{4},{7},{13},
				{14},{16},{6},{9},{12},{14},{16},{17},{2},{4},{11},{13},{15},{2},
				{7},{10},{14},{15},{17},{1},{3},{9},{15},{17},{3},{4},{5},{8},
				{15},{16},{1},{5},{7},{16},{17},{2},{4},{6},{8},{14},{2},{6},{7},
				{9},{13},{2},{3},{8},{9},{10},{11},{4},{5},{6},{7},{11},{12},{1},
				{7},{9},{10},{12}}},Pattern}]},
{VertexCoordinates->{{4,13},{4,6},{-0.2,5},{2,2.2},{-6,-1},{6,2.2},
		{9.5,1.6},{2.7,5},{4.8,7.8},{4,10.4},{-1.5,1.6},{14,-1},{4,3.6},
		{5.3,5},{3.2,7.8},{4,0.3},{8.1,5}}}]

GraphData["KittellGraph"] // InputForm

Graph[{1,2,3,4,5,6,7,8,9,10,11,12,13,14,15,16,17,18,19,20,21,
	22,23},{Null,SparseArray[Automatic,{23,23},0,
	{1,{{0,5,10,16,21,26,33,38,43,48,54,61,68,73,78,83,89,95,101,
				106,111,116,121,126},{{6},{11},{13},{17},{23},{3},{4},{11},{18},
				{20},{2},{12},{15},{16},{18},{20},{2},{8},{11},{14},{20},{7},{10},
				{17},{21},{23},{1},{11},{12},{13},{14},{19},{22},{5},{16},{17},
				{18},{21},{4},{14},{15},{20},{22},{10},{12},{16},{19},{21},{5},
				{9},{13},{19},{21},{23},{1},{2},{4},{6},{14},{17},{18},{3},{6},
				{9},{15},{16},{19},{22},{1},{6},{10},{19},{23},{4},{6},{8},{11},
				{22},{3},{8},{12},{20},{22},{3},{7},{9},{12},{18},{21},{1},{5},
				{7},{11},{18},{23},{2},{3},{7},{11},{16},{17},{6},{9},{10},{12},
				{13},{2},{3},{4},{8},{15},{5},{7},{9},{10},{16},{6},{8},{12},
				{14},{15},{1},{5},{10},{13},{17}}},Pattern}]},
{VertexCoordinates->{{10.5,3},{22,0},{10,18},{18,2.5},{5,3},
		{11.7,5.6},{1.8,1.5},{14,7},{7.9,9},{7.5,6},{12.8,1},{9,13},
		{9.5,6},{15,4},{11,13},{6.5,10},{7,1.8},{-1,0},{9.5,9},
		{13.5,10.5},{5.7,6.5},{12,8.5},{8,3.5}}}]
	
GraphData["SoiferGraph"] // InputForm

Graph[{1,2,3,4,5,6,7,8,9},{Null,SparseArray[Automatic,{9,9},0,
	{1,{{0,5,10,14,18,23,27,31,35,40},{{3},{4},{6},{7},{9},{4},{5},
				{6},{8},{9},{1},{5},{7},{8},{1},{2},{6},{8},{2},{3},{7},{8},{9},
				{1},{2},{4},{9},{1},{3},{5},{9},{2},{3},{4},{5},{1},{2},{5},{6},
				{7}}},Pattern}]},{VertexCoordinates->{{5.,0.},{2.5,2.35},{0.,0.},
		{3.2,3.},{1.8,1.7},{3.2,1.7},{1.4,0.6},{1.8,3.},{2.5,1.05}}}]
	
GraphData["FritschGraph"] // InputForm

Graph[{1,2,3,4,5,6,7,8,9},{Null,SparseArray[Automatic,{9,9},0,
	{1,{{0,5,9,13,18,23,28,32,37,42},{{2},{3},{4},{5},{6},{1},{4},
				{5},{8},{1},{4},{6},{9},{1},{2},{3},{8},{9},{1},{2},{6},{7},{8},
				{1},{3},{5},{7},{9},{5},{6},{8},{9},{2},{4},{5},{7},{9},{3},{4},
				{6},{7},{8}}},Pattern}]},{VertexCoordinates->{{0.8660254037844385,
			-0.5000000000000002},{0.,0.9999999999999999},{0.2635827842087346,
			-0.23527728656821076},{0.18827345317153082,0.029423848305956554},
		{-0.8660254037844388,-0.49999999999999944},{0.,-0.3235179376511209},
		{-0.2635677766589206,-0.23528526623224877},{-0.18826024207291525,
			0.029422289323712554},{0.,-0.14704541660158718}}}]


GraphData["PoussinGraph"] // InputForm

Graph[{1,2,3,4,5,6,7,8,9,10,11,12,13,14,15},
{Null,SparseArray[Automatic,{15,15},0,
	{1,{{0,5,10,15,21,25,30,36,40,45,50,56,62,68,73,78},{{3},{6},
				{11},{12},{13},{3},{6},{9},{10},{14},{1},{2},{6},{9},{12},{7},
				{8},{10},{13},{14},{15},{7},{11},{12},{15},{1},{2},{3},{10},{13},
				{4},{5},{9},{12},{14},{15},{4},{11},{13},{15},{2},{3},{7},{12},
				{14},{2},{4},{6},{13},{14},{1},{5},{8},{12},{13},{15},{1},{3},
				{5},{7},{9},{11},{1},{4},{6},{8},{10},{11},{2},{4},{7},{9},{10},
				{4},{5},{7},{8},{11}}},Pattern}]},
{VertexCoordinates->{{2.21,5.19},{2.19,5.4},{2.27,5.28},{1.98,5.57},
		{2.42,5.54},{2.11,5.28},{2.19,5.77},{1.61,5.07},{2.32,5.44},{2.07,
			5.44},{2.85,5.07},{2.61,5.19},{1.86,5.15},{2.19,5.53},{2.19,6.02}}}]