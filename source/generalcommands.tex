% https://tex.stackexchange.com/questions/410249/conflict-between-marvosym-and-mathabx
\newcommand{\MVAt}{{\usefont{U}{mvs}{m}{n}\symbol{`@}}}
\renewcommand{\spanishfigurename}{Figura}
\renewcommand{\spanishcontentsname}{Índice analítico}
\renewcommand{\listingscaption}{Programa}

% https://tex.stackexchange.com/questions/68080/beamer-bibliography-icon
\setbeamertemplate{bibliography item}{%
	\ifboolexpr{ test {\ifentrytype{book}} or test {\ifentrytype{mvbook}}
		or test {\ifentrytype{collection}} or test {\ifentrytype{mvcollection}}
		or test {\ifentrytype{reference}} or test {\ifentrytype{mvreference}} }
	{\setbeamertemplate{bibliography item}[book]}
	{\ifentrytype{online}
		{\setbeamertemplate{bibliography item}[online]}
		{\setbeamertemplate{bibliography item}[article]}}%
	\usebeamertemplate{bibliography item}}

\defbibenvironment{bibliography}
{\list{}
	{\settowidth{\labelwidth}{\usebeamertemplate{bibliography item}}%
		\setlength{\leftmargin}{\labelwidth}%
		\setlength{\labelsep}{\biblabelsep}%
		\addtolength{\leftmargin}{\labelsep}%
		\setlength{\itemsep}{\bibitemsep}%
		\setlength{\parsep}{\bibparsep}}}
{\endlist}
{\item}

\usetikzlibrary{calc,shapes}

%% styles for the hexboard
\tikzset{
	hexa/.style={ 
		shape=regular polygon,
		regular polygon sides=6,
		minimum size=1cm/sin(60),
		draw,
		inner sep=0,
		anchor=south,
		%%fill=#1,
		rotate=210,%%so ".corner 1" is at bottom of hexa shape
	},
	set hex board dimensions/.style={x=0.5cm, y=1cm*sin(60)},
	my hex path/.style={line width=2pt,line cap=round},
}

\makeatletter
\newif\if@surculus@shownodenames
\def\showmynodenames{\@surculus@shownodenamestrue}
\def\surculus@topcnt{4}
\def\surculus@botcnt{5}
\def\surculus@set@boardwidth#1{%%
	\def\surculus@botcnt{#1}%%
	\def\surculus@topcnt{\number\numexpr#1-1\relax}}

%% Here's a command to set up the board.  The argument 
%% is the width of widest row in the board
\def\drawhexboard#1{%%
	\surculus@set@boardwidth{#1}
	\tikzset{set hex board dimensions}
	\foreach \row in {1,...,\surculus@botcnt}{%
		\foreach \pos in {1,...,\row}
		{%%
			\node[hexa=red!20] (H\row;\pos) at (2*\pos-\row,\row) {};
			\if@surculus@shownodenames
			\node at (H\row;\pos) {\row;\pos};
			\fi
		}
	}
	\def\startcnt{\number\numexpr\surculus@botcnt+1\relax}
	\def\endcnt{\number\numexpr\surculus@topcnt+\surculus@botcnt\relax}
	\foreach \row in {\endcnt,...,\startcnt}{%
		\foreach \pos in {\endcnt,...,\row}{%
			\edef\surculus@rowpos{\number\numexpr\pos-\row+1\relax}
			\node[hexa=blue!10] (H\row;\surculus@rowpos) at (2*\pos-\row-2*\surculus@topcnt,\row) {};
			\if@surculus@shownodenames
			\node at (H\row;\surculus@rowpos) {\row;\surculus@rowpos};
			\fi
		}
	}
}

%% command to assist in drawing the path around the hexagons.
\def\drawhexpath(#1)#2{%%
	\let\surculus@previous\relax%%
	\foreach \x in {#2}
	{ \ifx\relax\surculus@previous
		\xdef\surculus@previous{\x}%%
		%% for testing %% \node [circle,fill=red,inner sep=2pt] at (H#1.corner \x) {};
		\else
		\draw[my hex path]  (H#1.corner \surculus@previous) -- (H#1.corner \x);
		\xdef\surculus@previous{\x}%%
		\fi
}}
\makeatother