% !TeX document-id = {e910d311-1b9f-412a-806f-acb4afe000c4}
% !TeX TXS-program:compile = txs:///pdflatex/[--shell-escape] | txs:///biber | txs:///pdflatex/[--shell-escape]

\documentclass[3p,times,a4paper,twocolumn,authoryear]{elsarticle} %authoryear

%% Fix so biblatex works instead of natbib
\makeatletter
\let\c@author\relax
\makeatother
\let\bibhang\relax
\let\citename\relax
\let\bibfont\relax
\let\Citeauthor\relax
\expandafter\let\csname ver@natbib.sty\endcsname\relax

%% Fix headers and footers
\makeatletter
\def\ps@pprintTitle{%
	\let\@oddhead\@empty
	\let\@evenhead\@empty
	\def\@oddfoot{\centerline{\thepage}}%
	\let\@evenfoot\@oddfoot}
\makeatother

%% Load some packages and stuff
%\usepackage[backend=biber,style=numeric, defernumbers=true, sorting=ynt,maxbibnames=4,maxcitenames=4]{biblatex}
%% Library and stuff
\usepackage[style=numeric,sortcites=true,sorting=ynt,backend=biber]{biblatex}
%% biblatex med apa-style. Det vivill ha
\DeclareLanguageMapping{american}{american-apa} %% Vi vill ha svensk apa
\bibliography{reference}

\usepackage{pifont}
\usepackage{geometry}
\usepackage{graphicx}
\usepackage{txfonts}
\usepackage{hyperref}
\usepackage{etoolbox}
\AtBeginDocument{%
	% \patchcmd{<cmd>}{<search>}{<replace>}{<success>}{<failure>}
	\patchcmd{\MaketitleBox}{\vspace*{-20pt}\fi}{\fi}{}{}%
	\patchcmd{\abstract}{Abstract}{Resumen}{}{}
}

\makeatletter
\patchcmd{\ps@pprintTitle}{\footnotesize\itshape
	Preprint submitted to \ifx\@journal\@empty Elsevier
	\else\@journal\fi\hfill\today}{\relax}{}{}
\makeatother

\usepackage{ecrc}
\volume{06}

\firstpage{10}
%\usepackage{etoolbox}
%\makeatletter
%\patchcmd{\ps@pprintTitle}% <cmd>
%{Preprint submitted}% <search>
%{To be submitted}% <replace>
%{}{}% <succes><failure>
%\makeatother
\runauth{Grupo N$^{\circ}6$}%C. Aznarán et al.

\jnltitlelogo{\large Annals of Discrete Mathematics FC-UNI}

\makeatletter

\def\keyword{%
	\def\sep{\unskip, }%
	\def\MSC{\@ifnextchar[{\@MSC}{\@MSC[2000]}}
	\def\@MSC[##1]{\par\leavevmode\hbox {\it ##1~MSC:\space}}%
	\def\PACS{\par\leavevmode\hbox {\it PACS:\space}}%
	\def\JEL{\par\leavevmode\hbox {\it JEL:\space}}%
	\global\setbox\keybox=\vbox\bgroup\hsize=\textwidth
	\normalsize\normalfont\def\baselinestretch{1}
	\parskip\z@
	\noindent\textit{Palabras clave: }  % <--- Edit as necessary
	\raggedright                         % Keywords are not justified.
	\ignorespaces}
\makeatother

%% `Elsevier LaTeX' style
%\bibliographystyle{elsarticle-num}
\renewcommand{\contentsname}{Tabla de contenidos}

\begin{document}

\begin{frontmatter}

\title{El teorema de los cuatro colores\tnoteref{t1}}
\tnotetext[t1]{This paper are available on \href{https://github.com/carlosal1015/4colores}{GitHub}.}
%% Group authors per affiliation:
\author[uni]{C. Aznarán Laos\fnref{myfootnote}}
\address{Facultad de Ciencias - Escuela profesional de Matemática}
\fntext[myfootnote]{Since 1880.}

\author[uni]{F. Cruz Ordoñez\corref{mycorrespondingauthor}}
\address{Facultad de Ciencias - Escuela profesional de Matemática}
\cortext[mycorrespondingauthor]{Corresponding author}
\ead{eduardodruz@gmail.com}

%% or include affiliations in footnotes:
\author[uni]{J. Navío Torres}
\address{Facultad de Ciencias - Escuela profesional de Ciencia de la Computación}
\ead[url]{www.blogdeoromion.pe.hu}

\author[uni]{G. Quiroz\corref{mycorrespondingauthor}}
\address{Facultad de Ciencias - Escuela profesional de Matemática}
\cortext[mycorrespondingauthor]{Corresponding author}
\ead{gabrielt@gmail.com}

\address[uni]{Universidad Nacional de Ingeniería,	Av. Túpac Amaru 210, Rímac, Lima 25, Peru}

%\author[uni]{Álvaro I. Plasencia\corref{mycorrespondingauthor}}
%\cortext[mycorrespondingauthor]{Corresponding author}
%\ead{caznaranl@uni.pe}\fnref{fn1}
%\ead[url]{www.blogdeoromion.pe.hu}\fnref{fn1}
%\author[uni]{C. Aznarán\fnref{fn2}}
%\fntext[fn1]{First author partially supported by the Universidad Nacional de Ingeniería project.}
%\fntext[fn2]{Second author partially supported by the Undergraduate Mathematics Group P156250.}
%
\begin{abstract}
In this paper, we will review the definitions about the hyperbolic functions, trigonometric functions and deduce their inverse functions in terms of logarithms and exponentials introducing complex numbers.
Hence, we will show the mathematical interrelation between this types of functions.
Having already present the established definitions for the hyperbolic functions will look for an analog representation for the trigonometric functions.% We will also give an introduction to inverse hyperbolic functions.
\\[0.2cm]
\textcopyright \hspace{.1cm}Science Department National University of Engineering Publishers Inc. All rights reserved. 
\end{abstract}

\begin{keyword}
four-color theorem \sep Kempe's chain\sep planar graphs
\end{keyword}

\end{frontmatter}

\tableofcontents

\section{Introducción}

\section{El ``camino'' hacia la demostración}

\section{Aplicaciones}

\section{Conclusiones}

\section*{Agradecimientos}

The authors want to thank the Universidad Nacional de Ingeniería and the Undergraduate Mathematics Group for their hospitality during the visits while preparing this paper.

The authors would like to thank professor Johny Valverde for many valuable and constructive suggestions, that have helped to improve the paper.

\nocite{*}
\printbibliography[title={Referencias}]
\end{document}
https://www.overleaf.com/17245402qmwcfgzsxdrd#/65670813/